%! suppress = Unicode
\documentclass{article}[12pt,a4paper]
\usepackage[utf8]{inputenc}
\usepackage{caption}
\usepackage{amsmath}
\usepackage{hyperref}
\usepackage{csquotes}

\hypersetup{
colorlinks=true,
linkcolor=blue,
filecolor=magenta,
urlcolor=cyan,
}

\title{NAHFPA specifikáció}
\author{Rátki Barnabás}
\date{2020.10.xx}

\newcommand{\lang}[1]{\textit{#1}}
\newcommand{\tbs}{\textbackslash}

\renewcommand*\contentsname{Tartalomjegyzék}

\begin{document}
    \maketitle

    \tableofcontents

    \section{Mi a program ?}\label{sec:mi-a-program-?}
    A NAHFPA = \textbf{NA}gy \textbf{H}ázi\textbf{F}eladat egy \textbf{PA}rser, program célja, hogy egy a \LaTeX -hez hasonló nyelvtanú
    nyelvben megírt matematikai kifejezést \textit{SVG} (Scalable Vector Graphics) formátummá fordítson.
    A felhasználó a program működését egy parancssoros (CLI) interfészen keresztül tudja irányítani.
    A következőkben először a leíró nyelv nyelvtanjáról lessz szó, utána pedig az interfész és egyébb támogató rendszerek működéséről.

    \section{Leíró nyelv}\label{sec:leíró-nyelv}
    ( A leíró nyelvet hívjuk az egyszerűség kedvéért, nahfpa-nak. )
    A formátuma egy egyszerű UTF-8 kódolású szöveges fájl aminek akármilyen kiterjesztése lehet, preferált a \textit{.nah}, vagy a \textit{.math}.
    Egy fájlban, vagy egységben, egy ábrát tudunk leírni, a nyelvnek nem célja a matematikai nyelvi helyességet ellenőrizni.
    A nyelv kifejezéseket tartalmaz, amik balról jobbra olvasva jelennek meg a megjelenítésben, egy kifejezés magában foglalhat gyakran másik kifejezéseket,
    a kifejezések leírása fog következni az ezen specifikációnak eleget tevő megjelenítési megkötésekkel és a kifejezések nyelvtani leírásával.
    A fordító helytelen nyelvtannal megírt fileokat, hibától függően két fajta módon kezel: a hibát képességi szerint ignorálja vagy a fordítást hibával végződik, fordított file nem keletkezik.
    ( A hibakezelés részletes leírása késöbb jön)

    \subsection{Kifejezésekből álltalánosan}\label{subsec:kifejezésekből-álltalánosan}
    A nyelvben vannak literálisok és parancsok.
    A literálisok csak be kell írni, és átalakítva megjelennek.
    A parancsok viszont a következő képpen épülnek fel: \lang{\tbs parancs\{paraméter1\}\{parméter2\}...} akárhány paraméterrel.
    A következőkben a \lang{[kif]} helyére akármilyen nyelvtani kifejezés behelyettesíthető.\\
    A \{ és \} jelekkel körül tudunk zárni kifejezéseket, hogy azok egy egymás mellé rakott kifejezésként viselkedjenek.
    Pl: \lang{\{a + b + c\}} kifejezés is beírható minden \lang{[kif]} helyére.
    Ez a nyelvtan hasonló lehet egy parancs parméterének megadására, de nem egyenértékűek, tehát például a: \lang{\tbs almaa} az nem az \lang{alma} parancs \lang{a} paraméterrel,
    viszont a \lang{\tbs alma\{\{a + b\}\}} parancsban az \lang{\{a + b\}} kifejezés lista a paraméter.

    \subsection{Szám literálok}\label{subsec:szám-literálok}
    Akármilyen csak numerikus karaktersorozat, ami tartalmazhat tizedes pontot vagy vesszőt\\
    Például: \lang{123456}, \lang{00120.232} mind úgy jelennek meg ahogy írjuk őket.
    
    \subsection{Szöveges literálok}\label{subsec:szöveges-literálok}


    \subsection{Egyszerű operátorok}\label{subsec:két-operandusu-egyszerű-operátorok}
    Pontosabban a \textit{+, -, *, /} jelek.
    Ezek akárhogyan elhelyezhetők illetve operandusuk elhagyhatóak. \\
    Például: \lang{[kif] + [kif]}, \lang{** [kif]} (két szorzás operátor egymás mellet aminél az elsőnek nincsen operandusa.) mind helyes kifejezések. \\
    \textbf{Megjelnítési megkötés:} A körbevevő kifejezésektől távolságot kell tartania.

    \subsection{}
    
\end{document}