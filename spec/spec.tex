%! suppress = Unicode
\documentclass{article}[12pt,a4paper]
\usepackage[utf8]{inputenc}
\usepackage{caption}
\usepackage{amsmath}
\usepackage{hyperref}
\usepackage{csquotes}
\usepackage{subfiles}

\hypersetup{
colorlinks=true,
linkcolor=blue,
filecolor=magenta,
urlcolor=cyan,
}

\title{NAHFPA specifikáció}
\author{Rátki Barnabás}
\date{2020.10.25}

\newcommand{\lang}[1]{\textit{#1}}
\newcommand{\tbs}{\textbackslash}
\newcommand{\tc}{\textasciicircum}

\renewcommand*\contentsname{Tartalomjegyzék}

\begin{document}
    \maketitle

    \tableofcontents

    \section{Mi a program ?}\label{sec:mi-a-program-?}
    A NAHFPA = \textbf{NA}gy \textbf{H}ázi\textbf{F}eladat egy \textbf{PA}rser, program célja, hogy egy a \LaTeX -hez hasonló nyelvtanú
    nyelvben megírt matematikai kifejezést \textit{SVG} (Scalable Vector Graphics) formátummá fordítson.
    A felhasználó a program működését egy parancssoros (CLI) interfészen keresztül tudja irányítani.
    A következőkben először a leíró nyelv nyelvtanjáról lessz szó, utána pedig az interfész és egyéb támogató rendszerek működéséről.

    \section{Leíró nyelv}\label{sec:leíró-nyelv}
    \subfile{sections/leironyelv}

    \section{Parancssori interfész}\label{sec:cli}
    \subfile{sections/cli}

    \section{Végszó}\label{sec:vég}
    A program implementációjának nem célja a lehető leggyorsabbnak, vagy legkissebb memóriaigényűnel lennie.\\
    Az első prioritás a programkód olvashatósága és fejleszthetősége.
    Futtatásához a bemeneti forráskód méretének háromszorosára van szüksége a fordítónak vagy 500nb-ra amelyik a nagyobb.
    A program ha létrehoz kimeneti fájlt annak tartalma mindenképpen nyelvtanilag helyes az SVG (https://www.w3.org/2000/svg) specifikáció szerint.

\end{document}