%! suppress = MissingLabel
%! suppress = Unicode
\documentclass{article}[12pt,a4paper]
\usepackage[utf8]{inputenc}
\usepackage{caption}
\usepackage{amsmath}
\usepackage{hyperref}
\usepackage{csquotes}
\usepackage{subfiles}

\hypersetup{
colorlinks=true,
linkcolor=blue,
filecolor=magenta,
urlcolor=cyan,
}

\title{Félkész Jelentés}
\author{Rátki Barnabás}
\date{2020.11.15}

\newcommand{\lang}[1]{\textit{#1}}
\newcommand{\tbs}{\textbackslash}
\newcommand{\tc}{\textasciicircum}

\begin{document}
    \maketitle
    
    \section{A program felépítése}
    Az egész forráskód az \lang{/src} mappában van.
    A programomat kisebb modulokból építettem fel, ezek mellett vannak adat struktúrák is külön implementálva.
    Ezek a pointereket tartalmazó linkelt lista, a \lang{List} és a dinamikus sztring a \lang{DString} (ez a későbbi még nem teljesen végleges, a későbbiekben C\# stílusura fogom átalakítani és nagyobb kapacitással fog rendelkezni mint a szükséges a fölösleges allokációk elkerülése végett.)
    A program részét képezi még a \lang{/utils} mappában levő segéd fileok, amik főleg definíciókat konstansokat és hasonló dolgokat tartalmaznak.
    Az alábbiakban részletezem a forrásfileok melyik részeit tartalmazzák a programnak (a feladatkiírás szerint):
    
    \subsection{data\_structures/DString}
    Dinamikus sztring implementácó a fentebb leírtak szerint.
    A struktúra nem teljes, ezzel a mezők el vannak rejtve a felhasználóktól, ez egy tudatos dizájn döntés.

    \subsection{data\_structures/List}
    Pointereket tartalmazó linkelt lista implementáció, definiál egy tisztán fejlesztési időben hasznos makrót a \lang{ListG}-t ami csak annyit tud, hogy a tipus meghatározásakor beleírhattam, hogy az adott list milyen tipusu pointereket tartalmaz.
    Ez az infó fordítás közben el van dobva.

    \subsection{expression\_parser/ExpNode}
    Ez a szubfolder a kifejezés feldolgozással foglalkozik, a kifejezés node a kifejezés fa adatstruktúrája, tartalmazza, hogy milyen argumentumai vannak a kifejezésnek, a tipusát és más hasznos információkat.
    Itt van definiálva a kifejezés fa naplózásához használható függvény.

    \subsection{expression\_parser/ExpParser}
    Ez a konkrét kifejezés feldolgozó modull.
    Ez használja a tokenizáló modult.
    Működésének kimenete a nyelvi kifejezés fa.
    
    \subsection{expression\_parser/TokenSlice}
    Egy segéd struktúra, ami definiálja a token listának egy indexekkel megadott szakaszát.
    
    \subsection{expression\_tokenizer/ExpTokenizer}
    Ezt a modult használja az ExpParser, egy egybefüggő sztringet vág szét használható szakaszokra.

    \subsection{logger/Logger}
    A naplózásért felelős modul, lehet megadni szinteket, illetve a létrehozásakor megadott konfiguráció szerint nem mindent ír ki, illetve fájlba is tud dolgozni.

    \subsection{nahfpa\_cli/NahfpaCli}
    A program főmodulja.
    Bemeneti a parancssoros argumentumok száma és értéke, ez hajta végre a fordítást és ehhez használja a kifejezés parszer modult és az svg gyártó modult.

    \subsection{svg\_factory/SVGFactory}
    A kifejezésfából először a box modell modul felhasználásával box modell fát, majd abból egy svg fájlt hoz létre.
    Az SvgFile modult használja a fájl létrehozására.

    \subsection{svg\_file/SvgFile}
    Ez a modul absztraktálja az svg generálás részleteit.

    \subsection{utils/cassert}
    Egy rövid függvényt definiál ami a standard assert kifejezés előtt kinaplózza a hibaüzenetet.

    \subsection{utils/geometry}
    A pont és méret struktúrákat és segédfüggvényeit tartalmazzák, az SvgFactory és SvgFile használják.

    \subsection{utils/help\_text}
    A \lang{--help} cli kapcsolóval elérhető segítség képernyő tartalmát definiálja.

    \subsection{utils/rendering\_constants}
    Svg megjelenítéshez szükséges konstansokat tartalmaz, például a betűk méretét.

    \subsection{utils/symbols\_helper}
    A támogatott szimbólumok definícióját és helyettesítési értékeit tartalmazza, egy segédfüggvénnyel ami megnézni, hogy az adott parancs milyen támogatott szimbólumot tartalmaz.

    \subsection{utils/test\_script}
    A program által jelenleg felhasznált bemenetet tartalmazza.

    \section{Összefoglaló}
    A program véleményem szerint a program több mint fele készenvan, sok nehéz részleten túlvagyok.
    A polírozás és több funkció implementálása még hátra van, de ezek már hasonlítanak elkészült dolgokhoz.
\end{document}