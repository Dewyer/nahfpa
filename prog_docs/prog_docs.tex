%! suppress = MissingLabel
%! suppress = Unicode
\documentclass{article}[12pt,a4paper]
\usepackage[utf8]{inputenc}
\usepackage{caption}
\usepackage{amsmath}
\usepackage{hyperref}
\usepackage{csquotes}
\usepackage{subfiles}

\hypersetup{
colorlinks=true,
linkcolor=blue,
filecolor=magenta,
urlcolor=cyan,
}

\title{Programozói Dokumentáció}
\author{Rátki Barnabás}
\date{2020.11.29}

\newcommand{\lang}[1]{\textit{#1}}
\newcommand{\tbs}{\textbackslash}
\newcommand{\tc}{\textasciicircum}

\newcommand{\fn}[1]{\paragraph{#1}\mbox{}\\}

\renewcommand*\contentsname{Tartalomjegyzék}

\begin{document}
    \maketitle

    \tableofcontents
    
    \section{A program felépítése}
    Az egész forráskód az \lang{/src} mappában van.
    A programomat kisebb modulokból építettem fel, ezek mellett vannak adat struktúrák is külön implementálva.
    A program részét képezi még a \lang{/utils} mappában levő segéd fileok, amik főleg definíciókat konstansokat és hasonló dolgokat tartalmaznak.

    \subsection{Általános design pattern}
    Az adatstruktúrák modulok nevei mind CamelCase-el vannak írva és példányosítandók a megfelelő \textbf{[nev]\_new} függvénnyel.
    \textbf{[nev]\_[függvény név]} alakú függvények mind az adott struktúrához való példányon dolgozó függvények, amiknek ha szükségük van rá első argomentumuk egy pointer a struktúrára aminek a kontextusába dolgoznak.
    Minden ilyenhez tartozik egy \textbf{[nev]\_free} függvény is, ez szabadítja fel az adatstruktúrát és minden általa tulajdonolt adatot (Nem minden adat ami része a kontextusnak tulajdona is).

    Az alábbiakban részletezem a forrásfileok melyik részeit tartalmazzák a programnak illetve ezeken belül mik a főbb függvények és azok mit csinálnak:
    
    \subsection{data\_structures/DString}
    Dinamikus sztring implementácó ami csak akkor hív újabb dinamikus foglalást ha elérte a capacitását ami 2-es alapú exponenciálisan nő. (C\# stílusu implementáció, könyebb mindig kicist több memóriát foglalni mint ami kell mint minden módosításnál új helyet foglalni)
    A struktúra nem teljes, ezzel a mezők el vannak rejtve a felhasználóktól, ez egy tudatos dizájn döntés.
    Továbbiakban CString elnevezés a c által használt null terminált karaktertömb pointer struktúrára utal.

    \fn{DString *DString\_from\_CString(const char *str)}
    Létrehoz egy új DString-et egy CString-ből másolással.

    \fn{DString *DString\_from\_CString(const char *str)}
    Visszatér az adott DString, CString alakú formájával.
    Ez az átalakítás kell a c standard könyvtárhoz való kompatibilitáshoz.

    \fn{size\_t DString\_len(const DString *self)}
    Visszatér a DString effektív hosszával karakterekben (null karaktereket nem számolva).

    \fn{void DString\_add\_char(DString *self, char chr)}
    Hozzáad a DStringhez egy karaktert.

    \fn{bool DString\_eq\_CString(const DString *string1, const char *string2)}
    Összehasonlít egy DStringet meg egy CStringet, igazzal tér vissza ha strcmp szerint egyenlőek.

    \fn{bool DString\_eq\_DString(const DString *string1, const DString *string2)}
    Összehasonlít két DString-et, igazzal tér vissza ha strcmp szerint egyenlőek.

    \fn{void DString\_concat(DString *self, const DString *string2)}
    Konkatanál két DString-et, úgy hogy az elsőt módosítja.

    \fn{void DString\_concat\_CString(DString *self, const char *string2)}
    Konkatanál egy DString-et meg egy Cstring-et úgy hogy az elsőt módosítja.

    \fn{char DString\_char\_at(const DString *self, size\_t ii)}
    Visszatér a DString \textit{ii}-edik karakterével.

    \fn{bool DString\_starts\_with(DString *self, char *string2)}
    Ellenőrzi, hogy az adott DString a megadott CString-el kezdődik-e.
    PL: almafa, és az alma bemenetekre igazat adna vissza.

    \fn{DString *DString\_substring(const DString *self, size\_t start, size\_t end)}
    Adott DString \textit{start} és \textit{end} indexei közötti részét adja vissza DString-ként. (Mind kettő index inklúzív)
    Asszerciós hibát dob ha akármelyik index is kint van a DString határain.

    \subsection{data\_structures/List}
    Pointereket tartalmazó linkelt lista implementáció, definiál egy tisztán fejlesztési időben hasznos makrót a \lang{ListG}-t ami csak annyit tud, hogy a tipus meghatározásakor beleírhattam, hogy az adott list milyen tipusu pointereket tartalmaz.
    Ez az infó fordítás közben el van dobva.

    \fn{void List\_push(List *self, void *item)}
    Hozzáfűz a listához egy pointert.

    \fn{void *List\_get(const List *self, size\_t index)}
    Visszatér a -edik elemmel a listában, ha nincsen ilyen elem null pointerrel tér vissza.

    \fn{void List\_free\_2D(List *self, void (free\_item)(void *))}
    Kétdimenziósan felszabadítja a listát, tehát felszabadítja a lista struktúráját és a tárolt adat pointert felszabadítja a megadott felszabadító függvénnyel.

    \fn{void List\_free(List *self)}
    Egydimenziósan felszabadítja a listát, tehát az adatpointerek által mutatott adatot nem szabadítja fel.

    \subsection{expression\_parser/TokenSlice}
    Egy segéd struktúra, ami definiálja a token listának egy indexekkel megadott szakaszát.

    \begin{center}
        \begin{tabular}{ | p{5.5cm} || p{5.5cm} | }
            \hline
            \textbf{Tipus / Név} & Leírás \\
            \hline
            \textbf{size\_t start} & Kezdő indexe a szakasznak\\
            \hline
            \textbf{size\_t end} & Vég indexe a szakasznak \\
            \hline
        \end{tabular}\\
        \textit{struct TokenSlice}
    \end{center}

    \fn{TokenSlice *TokenSlice\_shrink\_clone(TokenSlice *slice)}
    Lemásolja a szakaszt úgy, hogy a kezdő indexét megnöveli eggyel a végéből pedig levon egyet.

    \subsection{expression\_tokenizer/ExpTokenizer}
    Ezt a modult használja az ExpParser, egy egybefüggő sztringet vág szét használható szakaszokra.

    \fn{List *ExpTokenizer\_tokenize(ExpTokenizer *self)}
    Végrehajtja a tokenizálás a kontextusban megadott DString-en és visszaad egy DString pointer linkelt listát.
    A bemeneti nyers szövegen átmegy karakterenként, a feldolgozás alatt mindig van egy aktív buffer (\textit{DString *last\_token\_buffer}).
    Attól függően, hogy mi a jelenlegi karakter 3 dolog történhet.
    Ha a karakter whitespace akkor nem fűzzük hozzá a bufferhez és a hozzáadjuk a tokenek listájához majd kitöröljük a tartalmát (továbbiakban: flush).
    Ha a karakter egy egykarakteres operátor (Pl: *, +, -) akkor flush történik és utána fűződik hozzá a jelenlegi karakter a bufferhez majd ujra flush történik.
    Ha a karakter \textit{\textbackslash} akkor flusholunk és hozzáadjuk a jelenlegi karaktert a bufferhez.
    Ez a procedúra azt is elmenti egy listába, hogy az adott token utsó karaktere melyik indexen volt a nyers szövegben.

    \subsection{expression\_parser/ExpNode}
    Ez a szubfolder a kifejezés feldolgozással foglalkozik, a kifejezés node a kifejezés fa adatstruktúrája, tartalmazza, hogy milyen argumentumai vannak a kifejezésnek, a tipusát és más hasznos információkat.
    \begin{center}
        \begin{tabular}{ | p{5.5cm} || p{5.5cm} | }
            \hline
            \textbf{Tipus / Név} & Leírás \\
            \hline
            \textbf{ExpNodeType type} & A tipusa a kifejezés node-nak\\
            \hline
            \textbf{struct ExpNode *arg1} & Első pozícionális argomentum\\
            \hline
            \textbf{struct ExpNode *arg2} & Második pozícionális argomentum\\
            \hline
            \textbf{struct ExpNode *arg3} & Harmadik pozícionális argomentum\\
            \hline
            \textbf{ListG(struct ExpNode) *node\_list} & Linkelt node lista gyermekek\\
            \hline
            \textbf{DString *value} & Sztring érték a kifejezéshez, például a literál tartalma\\
            \hline
            \textbf{TokenSlice *origin} & Ezen a fa ezen részfája melyik részéből lett létrehozva a tokenlistának\\
            \hline
            \textbf{struct ExpNode *parent} & Node aminek a gyereke ez a node\\
            \hline
        \end{tabular}\\
        \textit{struct ExpNode}\\
        \textit{enum ExpNodeType { Index,
        NodeList,
        Symbol,
        Literal,
        Frac,
        Sqrt,
        Sum,
        Prod,
        Lim,
        Comment,
        Bracket}}
    \end{center}

    \fn{void ExpNode\_log(const ExpNode *self, Logger *logger)}
    Kinaplózzal a kifejezés node-ot a gyermekeivel együtt.

    \subsection{expression\_parser/ExpParser}
    Ez a konkrét kifejezés feldolgozó modul.
    Ez használja a tokenizáló modult.
    Működésének kimenete a nyelvi kifejezés fa.

    \fn{TokenSlice *ExpParser\_get\_bracketed\_slice(ExpParser *self, const DString *command, const DString *bracket, bool required, size\_t start\_i, size\_t max\_i)}
    Ez a függvény a token lista alapján egy adott parancshoz a megadott zárójel tipus (\textit{DString *bracket}) által közbezárt token lista részt.
    Ami a \text{start\_i} nél kezdődik és maximum \textit{max\_i}-ig tart.
    Ha a slice nem kötelező és nem található a megadott indexnél kezdődő pozíciótol a megfelelő zárójellel közbezárt rész akkor a függvény \textit{NULL}-al tér vissza.

    Ezt a függvényt használja mind a command argomentumok (\textit{\{}, \textit{\}}) mind a rendes zárójelek sliceját kiszámoló procedúrák.

    \fn{TokenSlice *ExpParser\_parse\_command\_args(ExpParser *self, ExpNode *command, int required\_args, int max\_args, size\_t args\_start, size\_t max\_i)}
    Egy ismert parancshoz parse-olja az argomentumait, a kötelező argomentumok (\textit{int required\_args}) számánál kevesebb mennyiségő argomentum, vagy maximumnál-nál (\textit{int max\_args}) több argomentum esetén hibába ütközik a fordítás.
    A feldolgozás az inklúzív \textit{size\_t args\_start} indexnél kezdődik.

    \fn{ExpNode *ExpParser\_parse\_command(ExpParser *self, DString *command, size\_t command\_start, size\_t max\_i)}
    A megadott parancs sztring (pl: \\frac), ból és a kezdő indexből visszatér a teljesen parseolt command-al.
    Csak támogatott command sztring lehet megadva.

    \fn{ExpNode *ExpParser\_parse\_node\_list(ExpParser *self, TokenSlice *slice)}
    Az egyik legfontosabb függvénye a modulnak, maga a gyökér kifejezés node is \textit{NodeList} tipusu.
    Ez indítja a rekurzív parsoló hívás sort, adva a token listában a konrkét slice amit fel kell dolgoznia, meghívja a további parsoló függvényeket, amik vagy ujrahívják a node lista feldolgozó függvényt, mert tartalmaznak újabb szinteket, vagy szimplán visszatérnek egy olyan ággal aminek nincsen több gyereke.

    \fn{ExpNode *ExpParser\_parse(ExpParser *self)}
    Elindítja az almodul segítségével a tokenizálást majd végrehajta a parseolást és visszatér a gyökér node pointerével.

    \subsection{logger/Logger}
    A naplózásért felelős modul, lehet megadni szinteket, illetve a létrehozásakor megadott konfiguráció szerint nem mindent ír ki, illetve fájlba is tud dolgozni.

    \fn{Logger *Logger\_new(LogLevel min\_level, char *out\_path, bool use\_color)}
    Példányosítja a modult, \textit{char *out\_path} lehet \textit{NULL} ha az akkor a kimenet a standard kimenetre kerül, ha meg van adva akkor a megadott elérési úton lévő fileba, ebben az esetben nincsenek szinek használva
    Ha \textit{bool use\_color} hamis akkor semmikép nincsenek szinek használva.

    \fn{void Logger\_log(Logger *self, LogLevel level, char *format, ...)}
    Végrehajta a naplózási infók kiírását a megadott fileba vagy a standard kimenetre a megadott naplózási szinten.
    Ha a minimum naplózási szintnél alacsonyabb szintű hívás történik akkor az nem fog semmilyen kimenetet létrehozni.

    \subsection{nahfpa\_cli/NahfpaCli}
    A program főmodulja.
    Bemeneti a parancssoros argumentumok száma és értéke, ez hajta végre a fordítást és ehhez használja a kifejezés parszer modult és az svg gyártó modult.

    \begin{center}
        \begin{tabular}{ | p{5.5cm} || p{5.5cm} | }
            \hline
            \textbf{Tipus / Név} & Leírás \\
            \hline
            \textbf{int arg\_count} & Parancssori argomentumok száma\\
            \hline
            \textbf{char **args} & Parancssori argomentumok kétdimenziós tömbje\\
            \hline
            \textbf{char *input\_file\_path} & Bemeneti file elérési útvonala *\\
            \hline
            \textbf{char *out\_file\_path} & Kimeneti SVG elérési útvonala *\\
            \hline
            \textbf{char *need\_color} & Szín kapcsolóba megadott szöveg tartalma *\\
            \hline
            \textbf{LogLevel min\_log\_level} & Minimum naplózási szint ami a Logger példánynak tovább lesz adva *\\
            \hline
            \textbf{Logger *pre\_logger} & Logger inicializálása előtt használt naplózási modul példány\\
            \hline
            \textbf{Logger *logger} & Összes függő modulnak továbbadott Logger singeleton *\\
            \hline
        \end{tabular}\\
        \textit{struct NahfpaCli}\\
        \textit{* -- Argomentum parsolás után feltöltött}
    \end{center}

    \fn{CliMode NahfpaCli\_parse\_args\_get\_cli\_mode(NahfpaCli *self)}
    Parszolja a kontextusban megadott cli argomentumokat, és betölti ez értékiket.
    Visszaadja a CliMode-ot is, ami azt indikálja, hogy a program fordítási módban lett elindítva, vagy megtalálható a \textit{help} flag, ami miatt csak az információs szöveg kerül kiírásra.

    \fn{DString *NahfpaCli\_read\_input(NahfpaCli *self)}
    Beolvassa a feldolgozandó szkriptet, vagy a standard bemenetről vagy a megadott fileból.
    Ha a kontextusban a beolvasandó file elérési útvonala \textit{NULL} akkor a standard bemenetről olvas.

    \subsection{svg\_factory/BoxModelBuilder}
    Ez a modul építí fel a kifejezés fa alapján a doboz fát aminek elemei a \textit{BoxNode}-ok.

    \begin{center}
        \begin{tabular}{ | p{5.5cm} || p{5.5cm} | }
            \hline
            \textbf{Tipus / Név} & Leírás \\
            \hline
            \textbf{Vector offset} & Relatív pozíció a szülőhöz képest\\
            \hline
            \textbf{Vector global\_pos} & Abszolút kordinátái a node-nak -- Csak a teljes fa felépítése után töltődik fel\\
            \hline
            \textbf{Size box} & Doboz mérete\\
            \hline
            \textbf{double centering\_axis\_y} & A doboz vertikális középre helyezéséhez használt relatív pozíció\\
            \hline
            \textbf{struct BoxNode *arg1\_box} & Első pozícionális argomentum doboza\\
            \hline
            \textbf{struct BoxNode *arg2\_box} & Második pozícionális argomentum doboza\\
            \hline
            \textbf{struct BoxNode *arg3\_box} & Harmadik pozícionális argomentum doboza\\
            \hline
            \textbf{ListG(struct BoxNode) *node\_list\_box} & Kifejezés lista dobozai\\
            \hline
            \textbf{ExpNode *node} & A kifejezés node amiből a doboz származik\\
            \hline
        \end{tabular}\\
        \textit{struct BoxNode}\\
    \end{center}

    \fn{BoxNode *BoxModelBuilder\_build\_box\_for\_node(BoxModelBuilder *self, ExpNode *exp\_node)}
    A modul legfontosabb függvénye, a rekurzív box fa építő hívás sor kezdete, a node tipusa alaján tovább hív másik függvényekbe amik már a renderelés részletei alapján kiszánolják az offsetet és a doboz méretét.

    \subsection{svg\_factory/AmfTextWidth}
    Ez a modul egy Helvetica karakter szélességét számítja ki egy adott magassághoz, egy AMF file tartalma van rögzítve egy tömbbe (\textit{SUPPORTED\_AMF\_CHARACTERS}).
    Optimalizációs céllal ebből a tömbből egy map van létrehozva, mivel tudjuk, hogy a kereshető karakterek egyértelműen leképezhetőek unsigned short ként.

    \fn{double AmfTextWidth\_char\_width\_calc(const AmfTextWidth *self, char cc, double font\_size)}
    Egy adott karakterhez és magassághoz kiszámítja a szélességet, használva a fent említett mappot.

    \subsection{svg\_file/SvgFile}
    Ez a modul absztraktálja az svg generálás részleteit.
    Megtartja az SVG működésének szabályait, a kordináta rendszer origója a bal felső sarok.
    Minden taghez egy meghatározott stílus osztályt használ, ezek a stílusok a \textit{const char *STYLE} globális változóban vannak definiálva.

    \fn{void SvgFile\_add\_text(SvgFile *self, const DString *text, const Vector *pos)}
    Létrehoz egy \textit{text} tag-et, a megadott kordinátáknál a szöveg magasságának korrigálásával (Tehát a szöveg bal felső sarka lesz a megadott kordináta mint minden más SVG objektumnál).
    A szöveg SVG escapelve van kiírás előtt.

    \fn{void SvgFile\_add\_path(SvgFile *self, const char* format, ...)}
    Létrehoz egy \textit{path} tag-et, a path argomentuma a tagnek egy varg formázható sztring (\textit{char *format}).

    \fn{void SvgFile\_add\_line(SvgFile *self, const Vector *p1, const Vector *p2)}
    Létrehoz egy \textit{line} tag-et, a megadott kordináták között.

    \fn{void SvgFile\_add\_box(SvgFile *self, const Vector *pp, const Size *size)}
    Létrehoz egy \textit{box} tag-et, a megadott kordinátánál a megadott mérettel.

    \subsection{svg\_factory/SvgFactory}
    A kifejezésfából először a box modell modul felhasználásával box modell fát, majd abból egy svg fájlt hoz létre.
    Az SvgFile modult használja a fájl létrehozására.

    \fn{void SvgFactory\_render\_node(SvgFactory *self, BoxNode *box\_node)}
    Kirenderel egy adott box node-ot a gyermekeivel együtt, meghív minden tipushoz egy segédfüggvényt.

    \subsection{utils/brackets}

    \fn{char *get\_closing\_bracket\_for\_bracket(const DString *bracket)}
    Segédfügvény ami egy adott zárójelhez visszaadja a záró párját.

    \subsection{utils/cassert}

    \fn{void cassert(Logger *log\_inst, int exp, char *msg, ...)}
    Egy rövid függvény ami a standard assert függvény meghívása előtt kinaplózza a hibaüzenetet.

    \subsection{utils/geometry}
    A pont és méret struktúrákat és segédfüggvényeit tartalmazzák, az SvgFactory és SvgFile használják.

    \subsection{utils/help\_text}
    A \lang{--help} cli kapcsolóval elérhető segítség képernyő tartalmát definiálja.

    \subsection{utils/rendering\_constants}
    Svg megjelenítéshez szükséges konstansokat tartalmaz, például a betűk méretét.

    \subsection{utils/symbols\_helper}
    A támogatott szimbólumok definícióját és helyettesítési értékeit tartalmazza, egy segédfüggvénnyel ami megnézni, hogy az adott parancs milyen támogatott szimbólumot tartalmaz.
    Bizonyos szimbólumoknak vannak nagybetűs verziói ezekhez speciális mezők vannak a szimbólum definíciókban.
    \begin{center}
        \begin{tabular}{ | p{5.5cm} || p{5.5cm} | }
            \hline
            \textbf{Tipus / Név} & Leírás \\
            \hline
            \textbf{char *command} & Parancs a szimbólumhoz\\
            \hline
            \textbf{char *substitution} & Helyettesítési érték\\
            \hline
            \textbf{char *uppercase\_substitution} & Nagybetűs helyettesítési érték\\
            \hline
            \textbf{Size box} & Mérete a szimbólumnak\\
            \hline
            \textbf{Size box\_uppercase} & Nagybetűs méret\\
            \hline
        \end{tabular}\\
        \textit{struct SymbolDefinition}
    \end{center}

    \fn{SymbolDefinitionFindResults SymbolDefinition\_get\_supported\_results(DString *command)}
    Megpróbálja megkeresni a parancsot a támagotatt szimbólumok tömbjéből, ellenőrizve azt is, hogy egy nagybetűs verzió-e.

    \begin{center}
        \begin{tabular}{ | p{5.5cm} || p{5.5cm} | }
            \hline
            \textbf{Tipus / Név} & Leírás \\
            \hline
            \textbf{bool found} & Megtaláta-e a keresés a támogatott szimbólumot\\
            \hline
            \textbf{const SymbolDefinition *definition} & NULL ha nem található, a szimbólum egyébbként a definícióra pointer\\
            \hline
            \textbf{bool is\_uppercase} & Keresett szimbólum egy nagybetűs verzió-e\\
            \hline
        \end{tabular}\\
        \textit{struct SymbolDefinitionFindResults}
    \end{center}

    \section{Test Suite}
    A programnak egy nem hivatalos része a test futtató program, csak azért létezik, hogy a fordító egy adott verzióján gyorsabb legyen kipróbálni sok példa scriptet.
    A futtatásához szükséges \textit{npm} és \textit{node}.
    Javascriptben íródott ezért nem is része a beadandómnak de az értékelőknek ajánlom mert sokat segíthet.

    \subsection{Használat}
    \textit{npm i} -- Telepíteni kell a függőségeket\\
    \textit{npm run test} -- Testek futtatása, a \textit{/test/test.html} a generált html file az eredményekkel\\
    \textit{npm run rast} -- Testek futtatása kimeneti SVG-k raszterizációjával az artifacts mappába

    \section{Összefoglaló}
    A program véleményem szerint a program több mint fele készenvan, sok nehéz részleten túlvagyok.
    A polírozás és több funkció implementálása még hátra van, de ezek már hasonlítanak elkészült dolgokhoz.
\end{document}