%! suppress = Unicode
\documentclass[../spec.tex]{subfiles}

\begin{document}

    A program egy parancssori interfészen keresztül használható.
    Ennek a működése parancsorban megadott argomentumokkal és zászlókkal irányítható.\\
    Példa a program használatára: \lang{nahfpa.exe -i input.math -o out.svg}\\

    \subsection{Naplózási szintek}\label{subsec:napló-szintek}
    A program működése közben naplózási adatokat ad ki, a standard kimenetre vagy a megadott naplózási fileba.
    A naplózásnak több szintje van:
    \begin{itemize}
        \item Infó : Információs adatok
        \item Warn : Figyelmeztetések
        \item Error : Hibák
    \end{itemize}
    Az adott naplózási információ tartalmától függ, hogy melyik szintben lesz kiadva.
    A felhasználó kimenetre küldött naplózást szűrheti, például: Csak figyelmeztetéseket vagy hibákat írjon ki a program.

    \subsection{Parancssori argumentumok}\label{subsec:args}
    \begin{itemize}
        \item -{}-input (-i) : Bemeneti fájl elérési útvonala | Standard bemenetről olvas be a program EOF jelig.
        \item -{}-out (-o) : Kimeneti fájl elérési útvonala | Alap érték: \lang{out.svg}
        \item -{}-log-level (-l) [szint] : A kimeneten megjelenő minimum naplózási szint beállítása, \lang{szint} lehetséges értékei: \lang{i, w, e} (Infó, Warning, Error) | Alap érték: Error
        \item -{}-log-file (-lf) [fájl] : A naplózái fájl elérési útvonala | Standard kimentre naplózik a program.
        \item -{}-color no : Kikapcsolja a szineket naplózásnál.
        \item -{}-help: Kiírja a program használati leírását, ha ez az argomentum használva van a program nem végez semmilyen fordítást és a többi argomentumot ignorálja.
    \end{itemize}
    | jel után lévő részek azt tartalmazzák, hogy mi történik ha a felhasználó nem adja meg az adott argomentumot.
    Zárójelben lévő verziói az argumentumoknak a rövidett változatok, ugyan úgy működnek mint a hosszabbak.
    
    \begin{center}
        Példa: \textit{nahfpa -i input.math -o out.svg -l warning}
    \end{center}

\end{document}